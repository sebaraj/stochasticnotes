\documentclass{article}
\usepackage{graphicx}
\usepackage{amsfonts}
\sepackage{amsmath}
\usepackage{amssymb}
\usepackage{enumitem}
\usepackage{xcolor}

\usepackage[a4paper, margin=1in]{geometry}


\title{S\&DS 351: Stochastic Processes - Homework 1}
\author{Bryan SebaRaj \\[0.8em] Professor Ilias Zadik}
\date{January 22, 2025}

\begin{document}

\maketitle

\section*{Problem 1.1}
Suppose you have a matrix $X \in \mathbb{R}^{m \times n}$ and another matrix $Y \in \mathbb{R}^{n \times p}$. Let $Z = X \times Y$, i.e., the matrix multiplication of $X$ and $Y$.

\begin{enumerate}[label=(\alph*)]
    \item (5 points) What are the dimensions of $Z$? What is the $i,j$th entry of $Z$ in terms of those of the matrices $X$ and $Y$? Is $Z$ necessarily equal to $Y \times X$? If not, provide a counterexample.

        \textcolor{blue}{The dimensions of $Z \in \mathbb{R}^{m \times p}$. \\ The $i,j$th entry of $Z$ is given by 
        $$Z_{i,j}=\sum_{k=1}^{n}A_{i,k}B_{k,j}$$ \\ 
         $Z$ is not necessarily equal to $Y \times X$. For example, consider the following matrices:
         $$X = \begin{bmatrix} 1 & 2 \\ 3 & 4 \end{bmatrix} \quad Y = \begin{bmatrix} 5 & 6 \\ 7 & 8 \end{bmatrix}$$
         Then, $Z = X \times Y = \begin{bmatrix} 19 & 22 \\ 43 & 50 \end{bmatrix}$, but $Y \times X = \begin{bmatrix} 23 & 34 \\ 31 & 46 \end{bmatrix}$.
     \\[0.6em] Therefore, there $\exists X, Y$ such that $X \times Y = Z \neq Y \times X$.}
    
    \item (5 points) Consider the following matrix $P$:
    \[
    P =
    \renewcommand{\arraystretch}{1.2}
    \begin{bmatrix}
        0 & 0 & 1 \\
        \frac{1}{3} & \frac{1}{3} & \frac{1}{3} \\
        0 & \frac{1}{2} & \frac{1}{2}
    \end{bmatrix}.
    \]
    Find $P^2$ (that is, $P \times P$).
    \renewcommand{\arraystretch}{1.2}
    \textcolor{blue}{$$P^2=P\times P = \begin{bmatrix}
        0 & 0 & 1 \\
        \frac{1}{3} & \frac{1}{3} & \frac{1}{3} \\
        0 & \frac{1}{2} & \frac{1}{2}
    \end{bmatrix} \begin{bmatrix}
        0 & 0 & 1 \\
        \frac{1}{3} & \frac{1}{3} & \frac{1}{3} \\
        0 & \frac{1}{2} & \frac{1}{2}
    \end{bmatrix}= \begin{bmatrix}
        0 & \frac{1}{2} & \frac{1}{2} \\
        \frac{1}{9} & \frac{5}{18} & \frac{11}{18} \\
        \frac{1}{6} & \frac{5}{12} & \frac{5}{12}
    \end{bmatrix}$$}

    \item (5 points) Find the limit of $P^n$ as $n \to \infty$ (that is, find the limit of each entry $(P^n)_{i,j}$, $1 \leq i,j \leq 3$ as $n \to \infty$). You do not need to prove what the limit is; it suffices to guess correctly (using a calculator or computer is allowed).

        \textcolor{blue}{Since $P$ is row-stochastic, $$ \lim_{x\to\infty} P^n=\begin{bmatrix} \pi \\ \pi \\ \pi\end{bmatrix}=\mathbf{1}\pi ^ {\top}$$
            where $\pi$ is the stationary distribution statisfying $\pi P = \pi$ and $\pi_1 + \pi_2 + \pi_3 = 1$. \\ 
        Let $\pi = (\pi_1, \pi_2, \pi_3)$. $$(\pi P)_j=\sum_{i=1}^{3}\pi_i P_{i,j}=\pi_j$$ for each $j=1,2,3$ \\ 
        For $j=1$, $\pi_1 = 0\pi_1 + \frac{1}{3}\pi_2 + 0\pi_3 = \frac{1}{3}\pi_2$ \\
        So, $\pi_1 = \frac{1}{3}\pi_2$ \\
        For $j=2$, $\pi_2 = 0\pi_1 + \frac{1}{3}\pi_2 + \frac{1}{2}\pi_3 = \frac{1}{3}\pi_2 + \frac{1}{2}\pi_3$, or $\pi_3=\frac{4}{3}\pi_2$ \\
        For $j=3$, $\pi_3 = \pi_1 + \frac{1}{3}\pi_2 + \frac{1}{2}\pi_3 = \pi_1 + \frac{1}{3}\pi_2 + \frac{1}{2}\pi_3$ \\
        Normalizing using $\pi_1 + \pi_2 + \pi_3 = 1$, $$\pi_1 + \pi_2 + \pi_3 = \frac{1}{3}\pi_2 + \pi_2 + \frac{4}{3}\pi_2=\frac{8}{3}\pi_2=1$$ \\
    Therefore, $\pi_2 = \frac{3}{8}$, $\pi_1 = \frac{1}{3}\pi_2 = \frac{1}{8}$, and $\pi_3 = \frac{4}{3}\pi_2 = \frac{1}{2}$. \\
$$\pi=\left( \frac{1}{8}, \frac{3}{8}, \frac{1}{2} \right)$$ 
     In a $3 \times 3$ matrix, where every row is $\pi$, we get 
     \[
\lim_{n \to \infty} P^n = 
\begin{bmatrix} 
    \frac{1}{8} & \frac{3}{8} & \frac{1}{2} \\ 
    \frac{1}{8} & \frac{3}{8} & \frac{1}{2} \\ 
    \frac{1}{8} & \frac{3}{8} & \frac{1}{2} 
\end{bmatrix}
\]}


\item (Bonus, 10 points) Prove the following statement for any $P \in \mathbb{R}^{3 \times 3}$. Assume the limit of $P^n$ as $n \to \infty$ equals a matrix of the form $\mathbf{1}\pi^\top$ for some $\pi \in \mathbb{R}^{3 \times 1}$ and $\mathbf{1} = (1, 1, 1)^\top \in \mathbb{R}^{3 \times 1}$. Confirm that $\mathbf{1} \pi^\top \in \mathbb{R}^{3 \times 3}$. Prove that $P^\top \pi = \pi$.

    \textcolor{blue}{Given that $\mathbf{1}=(1,1,1)^\top \in \mathbb{P}^{3\times 1}$ and $\pi \in \mathbb{R}^{3\times1}$, $\pi^\top$ is a $1\times 3$ matrix and $\mathbf{1}\pi^\top$ yields the dimensions $(3\times 1)\times (1\times 3)=3\times 3$. This confirms that $\mathbf{1}^\top\pi \in \mathbb{R}^{3\times 3}$ \\ \\ 
    Proving $P^\top \pi = \pi$: \\ \\
Given that $P^n \to \mathbf{1}\pi^\top$ as $n \to \infty$, $$P^{n+1}=P^n P \to (\mathbf{1}\pi^\top) \text{ as } n \to \infty$$
Since $P^{n+1}$ also converges to $\mathbf{1}\pi^\top$, $$\mathbf{1}\pi^\top = \mathbf{1}(\pi^\top P)$$
Since $\mathbf{1}$ is non-zero, this yields $$\pi^\top = \pi^\top P$$
Therefore, $\pi^\top$ is a left-eigenvector of $P$ with eigenvalue $1$; we can transpose both side to yield $$\pi=P^\top \pi$$
Thus, $P^\top \pi = \pi$}

        % \textcolor{blue}{Since $P$ is row-stochastic and the property is preserved under matrix multiplication, $$P^n\mathbf{1}=\mathbf{1}, \ \forall n$$
    %     Therefore, $$\lim_{n \to \infty} P^n\mathbf{1}=\mathbf{1}$$
    % Given that $\lim_{n \to \infty} P^n = \mathbf{1}\pi^\top$, $$\mathbf{1}\pi^\top\mathbf{1}=\mathbf{1}(\pi^\top\mathbf{1})=\mathbf{1}$$
% Therefore, $\pi^\top\mathbf{1}=1$ and $\mathbf{1}^\top\pi=1$}


\end{enumerate}


\section*{Problem 1.2}
Suppose that we are given two geometric random variables $A_1$ and $A_2$ with parameter $p$ which are not necessarily independent. Let $\{B_1, B_2, \dots\}$ be a sequence of random variables independent of $A_1$ and $A_2$, such that each $B_i$ has mean $\mu$ and variance $\sigma^2$.

\begin{enumerate}[label=(\alph*)]
    \item (5 points) Compute $\mathbb{E}[A_1 + 300A_2]$.

    \textcolor{blue}{Given that $A_1$ and $A_2$ are geometric, even if they are not independet, the expectation of a sum of random variables is the sum of their expectations. As such
    $$\mathbb{E}[A_1 + 300A_2]=\mathbb{E}[A_1] + 300\mathbb{E}[A_2]=\frac{1}{p}+\frac{300}{p}=\frac{301}{p}$$}

    \item (5 points) Prove that $\mathbb{P}[A_1 + 300A_2 \geq 5000/p] \leq 0.1$.

        \textcolor{blue}{Employing Markov's inequality, where for any non-negative random variable $X$ and any $a>0$, $$\mathbb{P}[X \geq a] \leq \frac{\mathbb{E}[X]}{a}$$
        $$\mathbb{P}\left( A_1 + 300A_2 \geq \frac{5000}{p} \right) \leq \frac{\mathbb{E}[A_1 + 300A_2]}{\frac{5000}{p}}=\frac{\frac{301}{p}}{\frac{5000}{p}}=\frac{301}{5000}$$
    Since $\frac{301}{5000} < 0.1$, $\mathbb{P}[A_1 + 300A_2 \geq 5000/p] \leq \frac{301}{5000} < 0.1$, proving the inequality.}


    \item (10 points) Compute $\mathbb{E}[\sum_{i=1}^{A_1} B_i^2]$. (Hint: condition on $A_1$).

        \textcolor{blue}{From the given information, $$\mathbb{E}[B^2_i]=Var(B_i)+(\mathbb{E}[B_i])^2 = \sigma^2 + \mu^2$$
        Using the law of total expectation and conditioning on $A_1$, $$\mathbb{E}[\sum_{i=1}^{A_1} B_i^2]=\mathbb{E}[\mathbb{E}[\sum_{i=1}^{A_1} B_i^2|A_1]]=\mathbb{E}[A_1(\sigma^2 + \mu^2)]=(\sigma^2 + \mu^2)\mathbb{E}[A_1]$$
        Since $A_1 \sim $ Geometric$(p)$, $$\mathbb{E}[A_1]=\frac{1}{p}$$  
        Therefore, $$\mathbb{E}\left[ \sum_{i=1}^{A_1}B^2_i \right] = \frac{\sigma^2 + \mu^2}{p}$$}

\end{enumerate}

\section*{Problem 1.3}
Suppose that two teams play a best of 5 series. That is, whichever team wins 3 games is the winner of the series. Suppose that each game is played independently, and for each game team $A$ has a probability 0.7 of winning and team $B$ has a probability 0.3.

\begin{enumerate}[label=(\alph*)]
    \item (5 points) What is the probability that team $A$ wins the series?

    \textcolor{blue}{The probability that A wins the best of 5 series can be denoted by $X\sim$ Binomial$(n=5,p-0.7)$ \\
        Therefore, $$\mathbb{P}(X \geq 3)=\sum_{k=3}^{5}\binom{5}{k}0.7^k0.3^{5-k}$$
        $$\mathbb{P}(X \geq 3)=\binom{5}{3}(0.7)^3(0.3)^2 + \binom{5}{4}(0.7)^4(0.3) + \binom{5}{5}(0.7)^5$$
        $$\mathbb{P}(X \geq 3)=10(0.7)^3(0.3)^2 + 5(0.7)^4(0.3) + (0.7)^5$$
        $$\mathbb{P}(X\geq 3)\approx 0.8369$$}

    \item (5 points) What is the probability that team $A$ wins the series conditioned on the fact that team $B$ won the first game? If you had to bet, would you bet on $A$ winning the series? Would you still bet on $A$ winning after $B$ won the first game?

        \textcolor{blue}{Given that $B$ won the first game, the series is now a best of 4 series, where $A$ needs to win 3 games, from the perspective of team $A$. Let $Y\sim$ Binomial$(n=4,p=0.7)$. Therefore, $$\mathbb{P}(Y \geq 3)=\sum_{k=3}^{4}\binom{4}{k}(0.7)^k(0.3)^{4-k}$$
        Therefore, $\mathbb{P}(A$ winning the series $| B$ won the first game$)=\mathbb{P}(Y \geq 3)=\binom{4}{3}(0.7)^3(0.3)+\binom{4}{4}(0.7)^4= 0.6517$ \\ 
        Given those probabilities, I would bet on $A$ winning the series, regardless if $B$ wins the first game.} 

\end{enumerate}

\section*{Problem 1.4}
Let $X, Y$ be two \textit{independent} standard normal random variables. Let $R$ be an exponential random variable with parameter 1 and let $\Theta$ be a uniform random variable taking values between $[0, 2\pi]$.

\begin{enumerate}[label=(\alph*)]
    \item (5 points) Compute $\mathbb{P}(R = 0)$ (please note that this isn’t the PDF of $R$ at 0, we are asking what is the probability that $R$ equals 0).
    
        \textcolor{blue}{Given that $R$ is an exponential random variable with parameter 1, its distribution is continuous on $[0,\infty)$. A fundamental property of continuous distributions is that they place 0 probability mass on a single point, or 
        $$\mathbb{P}(R=0)=\int_{0}^{\infty}f_R(r)\mathbf{1}{r=0}dr=0$$ where $f_R(r)=e^{-r} \ \forall r \geq 0$.
        Therefore, $\mathbb{P}(R=0)=0$}

    \item (10 points) Compute $\mathbb{P}(X^2 + Y^2 \geq t)$.

        \textcolor{blue}{Since $X,Y$ are independent $N(0,1)$, $X^2+Y^2$ follows a $\chi^2$ distribution with 2 degrees of freedom, 
            $$X^2 + Y^2 \sim \chi_{2}^2$$
            which is an exponential distribution with rate $\frac{1}{2}$
            $$X^2 + Y^2 \sim \text{Exp}\left(\frac{1}{2}\right)$$
            Specifically,
            $$\mathbb{P}(X^2+Y^2 \leq r)=1-e^{-\frac{r}{2}}, \\ \forall r \geq 0$$
            Therefore, 
            $$\mathbb{P}(X^2+Y^2 \geq t)=1-[1-e^{-\frac{t}{2}}]=e^{-\frac{t}{2}}$$
        $$\mathbb{P}(X^2+Y^2 \geq t)=e^{-\frac{t}{2}}, \ t \geq 0$$}

    \item (10 points) Assume that $R$ and $\Theta$ are independent. Define $A = \sqrt{R}\cos(\Theta)$ and $B = \sqrt{R}\sin(\Theta)$, what is the joint PDF of $A, B$? What is the marginal PDF of $A$?

        \textcolor{blue}{Calculating the joint PDF of $A,B$, we can first invert the transformation from $$A=\sqrt{R}\text{cos}\theta \text{ and } B= \sqrt{R}\text{sin}\theta$$
        to yield $R$ and $\Theta$ in terms of $A$ and $B$:
    $$R=A^2+B^2 \text{ and } \theta=\text{arctan}(B,A)$$
    This yields a bijection from $(r,\theta)\in [0,\infty) \times [0,2\pi] \text{ to } (a,b)\in \mathbb{R}^2$
Computing the Jacobian of the transformation using the generation formula for transformation of PDFs:
$$f_{A.B}(a,b)=f_{R,\Theta}(r,\theta)\cdot \left| \text{det}\frac{\partial(r,\theta)}{\partial(a,b)}\right|$$
First, note that $$f_{R,\Theta}(r,\theta)=f_R(r)f_{\Theta}(\theta)=(e^{-r}\mathbf{1}_{r\geq 0})(\frac{1}{2\pi}\mathbf{1}_{[0,2\pi]}\theta)=\frac{1}{2\pi}e^{-r}, \\ (r\geq 0, 0 \leq \theta \leq 2\pi)$$
Computing the forward Jaconbian, $$\frac{\partial a}{\partial r}=\frac{1}{2\sqrt{r}}\text{cos}\theta$$
$$\frac{\partial a}{\partial \theta}=-\sqrt{r}\text{sin}\theta$$
$$\frac{\partial b}{\partial r}=\frac{1}{2\sqrt{r}}\text{sin}\theta$$
$$\frac{\partial b}{\partial \theta}=\sqrt{r}\text{cos}\theta$$
Hence, $$\text{det}\frac{\partial (a,b)}{\partial (r,\theta)}=\frac{1}{2\sqrt{r}}\text{cos}\theta \cdot \sqrt{r}\text{cos}\theta - (-\sqrt{r}\text{sin}\theta \cdot \frac{1}{2\sqrt{r}}\text{sin}\theta)=\frac{1}{2}\text{cos}^2\theta + \frac{1}{2}\text{sin}^2\theta=\frac{1}{2}$$
Therefore, $$\text{det}\frac{\partial (r, \theta)}{\partial (a, b)}=\frac{1}{\frac{1}{2}}=2$$
Substituting $r, \theta$ and the Jacobian factor into $f_{R,\Theta}(r\theta)$:
$$f_{A,B}(a,b)=f_{R,\Theta}(r,\theta)(a^2+b^2, \text{ arctan}(b,a)) \times 2$$
Since $f_{R,\Theta}(r,\theta)=\frac{1}{2\pi}e^{-r}$, $$f_{A,B}(a,b)=\frac{1}{2\pi}e^{-(a^2+b^2)}\times 2=\frac{1}{\pi}e^{-(a^2+b^2)}, \ \forall a,b \in \mathbb{R}$$
Confirming that this integrates to 1 (i.e. a valid PDF)), $$\int_{\mathbb{R}}\frac{1}{\pi}e^{-(a^2+b^2)}da \ db = \frac{1}{\pi}\int_{0}^{2\pi} \int_{0}^{\infty}e^{-\rho^2}\rho d\rho \ d\theta = 2\int_{0}^{\infty}\rho e^{-\rho^2}d\rho$$
Setting $u=\rho^2$, $du=2\rho d\rho$, yield $$2\int_{0}^{\infty}\rho e^{-\rho^2}d\rho=2\frac{1}{2}=1$$
Therefore, the joint PDF of $A,B$ is valid.
The joint PDF of $A,B$ is $$f_{A,B}(a,b)=\frac{1}{\pi}e^{-(a^2+b^2)}, \ (a,b)\in \mathbb{R}$$
To find the marginal PDF of $A$, $B$ must be integrated out, $$f_A(a)=\int_{-\infty}^{\infty}f_{A,B}(a,b)db=\int_{-\infty}^{\infty}\frac{1}{\pi}e^{-(a^2+b^2)}db=\frac{1}{\pi}e^{-a^2}\int_{-\infty}^{\infty}e^{-b^2}db$$
Since $\int_{-\infty}^{\infty}e^{-b^2}db=\sqrt{\pi}$, $$f_A(a)=\frac{1}{\pi}e^{-\a^2}\times \sqrt{\pi}=\frac{1}{\sqrt{\pi}}e^{-a^2}$$
The marginal PDF can be demonstrated to be a valid PDF through integration: $$\int_{-\infty}^{\infty}\frac{1}{\sqrt{\pi}}e^{-a^2}da = \frac{1}{\sqrt{\pi}}\sqrt{\pi}=1$$
Therefore, the marginal PDF is a valid PDF.
Thus, the marginal PDF of $A$ is $$f_A(a)=\frac{1}{\sqrt{\pi}}e^{-a^2}, \ a \in \mathbb{R}$$}

\end{enumerate}

\section*{Problem 1.5}
Suppose that $X \sim \text{Exp}(\lambda_1)$, $Y \sim \text{Exp}(\lambda_2)$, and $Y$ is independent of $X$.

\begin{enumerate}[label=(\alph*)]
    \item (5 points) Compute $\mathbb{P}(X > Y)$.

        \textcolor{blue}{From the suppositions above, $$f_X(x)=\lambda_1e^{-\lambda_1x},\ x \geq 0$$ and $\mathbb{P}(X>t)=e^{-\lambda_1t}$ and 
        $$f_Y(y)=\lambda_2e^{-\lambda_2y},\ y \geq 0$$ and $\mathbb{P}(Y>t)=e^{-\lambda_2t}$. \\ \\ 
        $Y$ being independent of $X$ implies that $$f_{X,Y}(x,y)=f_X(x)f_Y(y), \forall x,y \geq 0$$
    $$\mathbb{P}(X>Y)=\int_{0}^{\infty}\int_{0}^{\infty}\mathbf{1}_{x>y}f_{X,Y}(x,y)dx\ dy$$
    $$\mathbb{P}(X>Y)=\int_{0}^{\infty}\int_{x=y}^{\infty}\lambda_1e^{-\lambda_1x}\lambda_2 e^{-\lambda_2 y} dx\ dy= \int_{0}^{\infty}e^{-\lambda_1y}\lambda_2 e^{-\lambda_2 y}\ dy$$
$$\mathbb{P})(X>Y)=\lambda_2 \int_0^{\infty}x^{-(\lambda_1 + \lambda_2)y}dy$$
$$\mathbb{P}(X>Y)=\lambda_2 \left( - \left. \frac{e^{-(\lambda_1 + \lambda_2)y}}{\lambda_1 + \lambda_2}\right|_{0}^{\infty} \right) $$
$$\mathbb{P}(X>Y)=\frac{\lambda_2}{\lambda_1 + \lambda_2}$$}

    \item (5 points) Compute $\mathbb{P}(X > (t + x) \mid X > t)$, for $t > 0$ and $x > 0$.

    \textcolor{blue}{From the definition of conditional probability, $$\mathbb{P}(X > (t + x) \mid X > t)=\frac{\mathbb{P}(X > (t + x) \cap X > t)}{\mathbb{P}(X > t)}$$
        Since $x>0$, ${X>t+x}$ is contained in ${X>t}$, so $\mathbb{P}(X > (t + x) \cap X > t)=\mathbb{P}(X > t + x)$ \\ \\ 
    Therefore, $$\mathbb{P}(X > (t + x) \mid X > t)=\frac{\mathbb{P}(X > t + x)}{\mathbb{P}(X > t)}$$
Since $X \sim $ Exp$(\lambda_1)$, $$\mathbb{P}(X>t+x|X>t)=e^{\lambda_1(t+x)},\text{ and }  \mathbb{P}(X>t)=e^{-\lambda_1t}$$
Therefore, $$\mathbb{P}(X>t+x|X>t)=\frac{e^{-\lambda_1(t+x)}}{e^{-\lambda_1t}}=e^{-\lambda_1x}$$}

    \item (5 points) Compute $\mathbb{P}(\min(X, Y) > t)$.

        \textcolor{blue}{${min(X,Y)>t}$ can be rewritten as ${X>t \cup Y>t}$. \\ \\ Therefore, given that $X$ and $Y$ are independent, $$\mathbb{P}(\min(X,Y)>t)=\mathbb{P}(X>t)\mathbb{P}(Y>t)$$
            Since $X \sim$ Exp$(\lambda_1)$ and $Y \sim$ Exp$(\lambda_2)$, $$\mathbb{P}(X>t)=e^{-\lambda_1t} \text{ and } \mathbb{P}(Y>t)=e^{-\lambda_2t}$$
        Therefore, $$\mathbb{P}(\min(X,Y)>t)=e^{-\lambda_1t}e^{-\lambda_2t}=e^{-(\lambda_1+\lambda_2)t}$$}

\end{enumerate}

\section*{Problem 1.6}
Given a fair die with 8 possible sides, let $T$ be the number of times you have to roll so that all eight sides have appeared at least once. Let $N$ be the number of distinct sides obtained from the first eight rolls.

\begin{enumerate}[label=(\alph*)]
    \item (5 points) Find $\mathbb{E}(T)$.

        \textcolor{blue}{Let $t_i$ be the number of die rolls required to obtain the $i$th distinct side, where $i=1,2,...,8$, after $i-1$ distinct rolls have been observed.
            $$T=t_1+t_2+...+t_n$$
            Note that the probability of rolling a distinct side on the $i$th roll is $\frac{n-i+1}{n}$. Therefore, $t_i$ has a geometric distribution with expectation $$\frac{1}{p_i}=\mathbb{E}[t_i]=\frac{n}{n-i+1}$$
            By linearity of expectations, $$\mathbb{E}[T]=\mathbb{E}[t_1+t_2+...+t_n]$$
            $$\mathbb{E}[T]=\mathbb{E}[t_1]+\mathbb{E}[t_2]+...+\mathbb{E}[t_n]$$
            $$\mathbb{E}[T]=\frac{n}{n}+\frac{n}{n-1}+...\frac{n}{1}$$
            $$\mathbb{E}[T]= 8\left(\frac{1}{8} + \frac{1}{7} + \frac{1}{6} + \frac{1}{5} + \frac{1}{4} + \frac{1}{3} + \frac{1}{2} + 1 \right)$$
$$\mathbb{E}[T]\approx 21.743$$}

    \item (5 points) Find $\mathbb{E}(N)$.

    \textcolor{blue}{For $1\leq i \leq 8$, let $X_i$ be the indicator random variable which equals 1 if $i$th distinct side appears at least once in the first eight rolls, and 0 otherwise. \\ \\ 
    The total number of distinct sides obtained is then $$N=\sum_{i=1}^{8}X_i$$
By linearity of expectations, $$\mathbb{E}[N]=\mathbb{E}[\sum_{i=1}^{8}X_i]=\sum_{i=1}^{8}\mathbb{E}[X_i]$$
Since $X_i$ is an indicator random variable, $\mathbb{E}[X_i]=\mathbb{P}(X_i=1)$, which is the probability that the $i$th distinct side appears at least once in the first eight rolls. \\ \\
$$\mathbb{P}(X_i=1)=1-\mathbb{P}(\text{face $i$ does not appear in the first eight rolls})$$
$$\mathbb{P}(X_i=1)=1-\left(\frac{7}{8}\right)^8$$
Therefore, $$\mathbb{E}[N]=8\left[1-\left( \frac{7}{8} \right)^8 \right]$$
$$\mathbb{E}[N]\approx 5.251$$}


    \item (5 points) Find $\mathbb{E}(T \mid N = 4)$.

    \textcolor{blue}{$T$ can be expressed as $T=8+(T-8)$. Let $T'$ be the number of rolls required to see all missing, distinct sides, starting after roll 8 (i.e. on roll 9). $$T'=T-8$$ 
    $$\mathbb{E}[T | N = 4]=\mathbb{E}[8+(T-8)|N=4]=8+\mathbb{E}[T'|N=4]$$
    (Note: I'm going to attempt to define this as a Markov chain, but the same logic from part (a) can also be applied). \\ \\ 
    Let $S$ be a set of relevant states, where $S=\{0,1,2,3,4\}$, where state $i$ represents the number of missing, unobserved sides (e.g. state 0 represents the observation of all 8 distinct sides). \\ \\ 
    Let the Markov chain start in state 4 (since $8-N=8-4=4$) and let $p_{ij}$ be the probability of transitioning from state $i$ to state $j$, such that $j=i-1$. \\ \\
From any state $i$, $p(i,j)=\frac{i}{8}$. Therefore, from state $i$ where $1\leq i \leq 4$: $$\mathbb{P}(i,j)=\frac{i}{8} \text{ and } \mathbb{P}(i,i)=1-\frac{i}{8}$$
and for state 0, we remain in state 0 with probability 1, i.e. state 0 is the absorbing state. \\ \\
Let $T_i$ be the number of rolls required to see all missing, distinct sides, starting from state $i$ to state 0. From the markov chain, we can define the standard equation:
$$T_i=1+\frac{i}{8}T_{i-1}+\left(1-\frac{i}{8}\right)T_i$$
$$T_i=\frac{8}{i}+T_{i-1}$$
This takes the closed form: $$T_i=\sum_{k=1}^{i}\frac{8}{k} \text{ where } T_0=0$$
$$T_4 = 0 + \frac{8}{1} + \frac{8}{2} + \frac{8}{3} + \frac{8}{4} = \frac{50}{3}$$
Therefore, $$\mathbb{E}[T'|N=4]=\frac{50}{3}$$
Since $\mathbb{E}[T|N=4]=8+\mathbb{E}[T'|N=4]$, $$\mathbb{E}[T|N=4]=8+\frac{50}{3}=\frac{74}{3}$$}

\end{enumerate}


\end{document}

