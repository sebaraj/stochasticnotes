\documentclass{article}
\usepackage{graphicx}
\usepackage{amsfonts}
\usepackage{amsmath}
\usepackage{amssymb}
\usepackage{enumitem}
\usepackage{xcolor}

\usepackage[a4paper, margin=0.5in]{geometry}



\title{S\&DS 351 - Stochastic Processes - Lecture 17 Notes}
\author{Bryan SebaRaj}
\date{March 26, 2025}

\begin{document}

\maketitle

% \section{Section}

\includegraphics[width=0.9\textwidth]{../images/IMG_0730.jpg}

\includegraphics[width=0.6\textwidth]{../images/IMG_0731.jpg}
\includegraphics[width=0.4\textwidth]{../images/IMG_0732.jpg}

\includegraphics[width=0.6\textwidth]{../images/IMG_0733.jpg}
\includegraphics[width=0.4\textwidth]{../images/IMG_0734.jpg}


\subsection*{Martingale convergence theorem}

Martingales enjoy a very nice calculus property: They converge with probability
1 under a mild boundedness assumption.

Recall from calculus: 
$$X_1, ..., X_n \in \mathbb{R}$$
converges to a limit $x\in \mathbb{R}$

\noindent ($\lim_n x_n=x$), if $(x_n-x)$ small as $n\rightarrow +\infty$, if $\forall \epsilon>0$,$\exists n_0$ s.t.$n>n_0$, $|x_n-x|<\epsilon$.

\includegraphics[width=0.7\textwidth]{../images/IMG_0735.jpg}

\includegraphics[width=0.9\textwidth]{../images/IMG_0736.jpg}

\includegraphics[width=0.7\textwidth]{../images/IMG_0737.jpg}

\includegraphics[width=0.7\textwidth]{../images/IMG_0738.jpg}




\end{document}


