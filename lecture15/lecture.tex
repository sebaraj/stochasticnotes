\documentclass{article}
\usepackage{graphicx}
\usepackage{amsfonts}
\usepackage{amsmath}
\usepackage{amssymb}
\usepackage{enumitem}
\usepackage{xcolor}

\usepackage[a4paper, margin=0.5in]{geometry}



\title{S\&DS 351 - Stochastic Processes - Lecture 15 Notes}
\author{Bryan SebaRaj}
\date{March 5, 2025}

\begin{document}

\maketitle

% \includegraphics[scale=0.6]{../images/}

\subsection*{Submartingales and Supermartingales}
Definition: a process $X_n, n \geq 0$ is a submartingale (subMG) w.r.t. to a
process $W_n, n\geq0$, if $\forall n \in N, X_n \leq
\mathbb{E}[X_{n+1}|W_{0:n}]$ 

Definition: a process $X_n, n \geq 0$ is a supermartingale (superMG) w.r.t. to
a process $W_n, n\geq0$, if $\forall n \in N, X_n \geq
\mathbb{E}[X_{n+1}|W_{0:n}]$ 

Example: Random walk on $\mathbb{Z}$, biased (w.p. $p\in [0,1]$),

\includegraphics[width=0.4\textwidth]{../images/IMG_0668.jpg}
\includegraphics[width=0.5\textwidth]{../images/IMG_0669.jpg}


\subsection*{Optional Stopping Theorem (OST)}
Simple tower property:

For $M_n$ in a martingale, 

$$\mathbb{E}[M_{n+1}|W_{0:n}]=M_n$$

$$\mathbb{E}[\mathbb{E}[M_{n+1}|W_{0:n}]]=\mathbb{E}[M_n]$$

$$\mathbb{E}[M_{n+1}]=\mathbb{E}[M_n] = ... = \mathbb{E}[M_0]$$

$$\forall n \geq 0, \mathbb{E}[M_n]=\mathbb{E}[M_0]$$

Martingale is in fact fiar in a much stronger sense than (*) is T is a random time (some assumptions), then 
$$\mathbb{E}[M_T]=\mathbb{E}[M_0]$$


\includegraphics[width=0.8\textwidth]{../images/IMG_0670.jpg}

\subsection*{Remarks}

\begin{enumerate} 
    \item From MCs, we have seen stopping times, e.g. for a
stochastic random walk $S_n, n\geq 0$, $T_1 = inf\{n\geq 0 | S_n = 1\}$ or $T_2
= inf\{n\geq 0 | S_n = 1000 or S_n = -3\}$ are stoppign times \\
Note: $T_1 = \infty$ if $S_n \neq 1, \forall n \in \mathbb{N}$ \\
$\forall k = 0, 1, 2, ...$, $$1(T_1=k) = 1(S_0 \neq 1, S_1\neq 1, ..., S_{k-1}\neq 1, S_k =1$$ a function of $S_0, S_1, ..., S_k$ \\
\item not a stopping time? $S_n$ is a stochastic random walk $T_3 = \sup\{n\leq 3 | S_n  =\max_{0\leq k \leq 3} S_k\}$

    \includegraphics[width=0.8\textwidth]{../images/IMG_0671.jpg}

    In other words, cannot stop early, need to know the future to determine if $S_n$ is a stopping time.
\item Stopping times make sense as strategies of a gambler. At time $k$, a gambler wants to either stop and leave or continue playing.
    The information until time $n$ is $W_{0:n}$, $\{T=k\}, \forall k$ should be a function of $W_{0:k}$.

\end{enumerate}

\subsection*{OST}
For $T$ stopping time, $\mathbb{E}(M_T)= \mathbb{E}(M_0)$. But need $\mathbb{P}(T < +\infty)=1$

\subsection*{Example}

    \includegraphics[width=0.8\textwidth]{../images/IMG_0672.jpg}

    \includegraphics[width=0.8\textwidth]{../images/IMG_0673.jpg}
            
    \includegraphics[width=0.8\textwidth]{../images/IMG_0674.jpg}

\subsection*{Theorem}

\includegraphics[width=0.8\textwidth]{../images/IMG_0675.jpg}

\includegraphics[width=0.8\textwidth]{../images/IMG_0677.jpg}


    




\end{document}


